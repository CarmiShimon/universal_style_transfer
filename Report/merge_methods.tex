In their paper \cite{bib11}, Li et al describe an interpolation technique to generate texture from two input style images. According to their method, which we shall refer to as \textit{Original-Merge}, at each pipeline level two WCT calls are conducted: $f_{cs_1} = WCT(x, f_{s_1}), f_{cs_2} = WCT(x, f_{s_2})$ where $x$ is the content features output by the level encoder. The results are then interpolated with $f_{cs} = \beta f_{cs_1} + (1-\beta)f_{cs_2}$. After the interpolation, $f_{cs}$ is reconstructed as usual using the level decoder.\\

In this section we researched other merging techniques to merge two style with a \textbf{single} WCT call for each pipeline level. We present the following three techniques:

\begin{itemize}
	\item \textbf{Level Merge}: Let $P_1, \dots, P_k$ be the pipeline level functions. According to this notation, in Figure ~\ref{fig:full-pipeline} the output is a product of:
	\begin{equation*}
	I_{out} = P_5 ( P_4 ( P_3 ( P_2 ( P_1 (c,s),s) ,s),s), s)
	\end{equation*}
	To \textit{Level Merge} transfer content $c$ with style images $s_1$ and $s_2$ do the following:
	\begin{equation*}
	I_{out} = P_5 ( P_4 ( P_3 ( P_2 ( P_1 (c, s_1), s_2), s_1), s_2), s_1)
	\end{equation*}
	This method simply alternates the transferred style at each progression through the pipeline levels. In Level-Merge, the WCT procedure is called once per level, which makes it more computationally efficient than Original-Merge. Since in each level it calls for the encoder twice, the WCT once, and the decoder once, it has the same computational load of regular UST with one style image.
	 
	\item \textbf{Channel Merge}: Let $s_1, s_2$ be the input pair of styles, and let $c'$ be the content input to the $j$-th pipeline level. Under Channel Merge, we extract features for all three: $f_{s_1} = E_j(s_1), f_{s_2} = E_j(s_2), f_{c'} = E_j(c')$. Then, we split $f_{s_1}$ and $f_{s_2}$ for $d$ splits across the channel dimension to get $f_{s_1} = \begin{bmatrix} f_{s_1}^1 \\ \vdots \\ f_{s_1}^d\end{bmatrix}$ and $f_{s_2} = \begin{bmatrix} f_{s_2}^1 \\ \vdots \\ f_{s_2}^d\end{bmatrix}$. Then, create synthetic $\tilde{f_{s}}$ by alternatively stacking splits from $f_{s_1}$ and $f_{s_2}$. In our implementation we set  $d=4$, for which $\tilde{f_{s}} = \begin{bmatrix} f_{s_1}^1 \\  f_{s_2}^2 \\  f_{s_1}^3 \\ f_{s_2}^4\end{bmatrix}$. After that, the level output is achieved by $f_{c',s_1,s_2} = WCT(f_{c'}, \tilde{f_{s}})$, and then $c'' = D_j(f_{c',s_1,s_2})$. In Channel Merge, in each pipeline level, still a single WCT call is conducted, but three feature extractions are made, making it slightly more computationally heavy than the Level Merge method. A limitation to this technique is the necessity for the pair of style to have same dimensions, this can be achieved simply by resizing the style inputs.
	
	\item \textbf{Interpolate-Style Merge}: As described above, Original Merge conducts \textbf{two} WCT calls in each pipeline level and interpolates their results, before feeding it to the level decoder. Relating to this, in Interpolate-Style Merge, the extracted features of both styles are interpolated \textbf{before} the WCT call, which operates on the features of the content and the interpolated features of both styles together. Formally, to calculate $j$-th pipeline level output, $c_j$, do:
	\begin{enumerate}
		\item Input $c_{j-1}, s_1, s_2$
		\item Extract features: $f_{c_{j-1}} = E_j(c_{j-1})$, $f_{s_1} = E_j(f_{s_1})$, $f_{s_2} = E_j(f_{s_2})$
		\item Interpolate style features: $\tilde{f_{s}} = \beta\cdot f_{s_1} + (1-\beta)\cdot f_{s_2}$
		\item Run WCT: $f_{cs} = WCT(c_{j-1}, \tilde{f_{s}})$
		\item Output decoded $f_{cs}$: $c_j = D_j(f_{cs})$
	\end{enumerate}
	In this method too, in each pipeline level, still a single WCT call is conducted, and three feature extractions are made, making it slightly more computationally heavy than the Level Merge method. The same limitation from Channel Merge applies to this technique as well, is the necessary for the pair of style to have same dimensions, which is done, again, by resizing the style inputs.
\end{itemize}
These three methods are implemented in the function \texttt{merge\_function} in file \texttt{utils.py}.
%show an improved algorithm which merges two style images bu using WCT algorithm which based on singular value decomposition (SVD). Here, we both implement the original merge algorithm as proposed in \cite{bib11} as well as introduce three additional efficient methods based on the use WCT.